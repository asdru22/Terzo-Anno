\documentclass[12pt,a4paper,oneside]{book}


\usepackage[utf8]{inputenc}
\usepackage[a4paper,inner=3.5cm,outer=2.5cm]{geometry}

\usepackage[titletoc,title,toc,page]{appendix}
\usepackage{verbatim}
\usepackage{placeins}
\usepackage{listings}
\usepackage{hyperref}
\usepackage[italian]{babel}
\usepackage{tikz}
\usepackage{parskip}
\usepackage{url}

\usepackage{graphicx}
\usepackage{blindtext}
\usepackage{chngcntr}
\counterwithin{table}{chapter}

\usepackage{newlfont}
\usepackage{fancyhdr}
\usepackage{indentfirst}
\usepackage[utf8]{inputenc}
\usepackage{float}
\usepackage{hyperref}
\usepackage[capitalize,noabbrev]{cleveref}
\usepackage{soul}
\usepackage[font=footnotesize,labelfont=bf]{caption}

\usepackage{multirow}
\usepackage{hyphenat}
\hyphenation{mate-mati-ca recu-perare}

\usepackage{lscape} 

\usepackage{titlesec}

\usepackage{natbib}
\bibliographystyle{alpha}
\setcitestyle{super,open={[},close={]}}
\usepackage{hyperref}
\newcommand{\rom}[1]{\uppercase\expandafter{\romannumeral #1\relax}}


\usepackage{etoolbox}
\makeatletter
\patchcmd{\chapter}{\if@openright\cleardoublepage\else\clearpage\fi}{}{}{}
\makeatother

\usepackage{pdfpages}

\begin{document}
% Per spostare i vari elementi più su o più giù gioca con i valori di vspace che ci sono tra uno e l'altro
\pagestyle{empty}
\newgeometry{
    left=20mm,
    right=20mm,
    top=20mm,
    bottom=20mm
}

\begin{titlepage}

\begin{center}

% marchio di ateneo
\includegraphics[width=6.5cm,height=4.7cm]{img/marchio-di-ateneo.png}

\vspace{15mm}


% \Large is 14.4pt
{\Large{\bf{Tesina Storia e Politiche del digitale}}}

\vspace{15mm}

{\Huge{\bf I costi materiali delle }}\\
\vspace{3mm}
{\Huge{\bf tecnologie del XXI secolo}}\\
\vspace{3mm}
\end{center}

\vspace{20mm}

\hfill
\begin{minipage}[t]{0.40\textwidth}\raggedleft
{\Large{\bf Presentata da: \\ Alessandro Nanni \\ mat. 
 0001027757}}
\end{minipage}

\vspace{90mm}

\rule[0.5cm]{15.8cm}{0.6mm}

\begin{center}
{\large{\bf Anno Accademico 2024/2025\\}}
\end{center}

\end{titlepage}

\restoregeometry


\pagenumbering{gobble}


\topmargin=-2cm
\setcounter{page}{1}
\textheight=25cm  % Decrease this value to increase bottom margin
\tableofcontents

\pagenumbering{arabic}
\raggedbottom

\setcounter{chapter}{-1}
\pagestyle{plain}
\chapter{Introduzione}

Se si considera il lavoro dietro a uno smartphone, un sito web o piattaforme di chatbot basate su intelligenza artificiale come ChatGPT, vengono subito in mente le figure degli sviluppatori software e degli ingegneri informatici. Tuttavia, ci sono fattori che, pur essendo altrettanto rilevanti, sono spesso nascosti o sottovalutati, e che meritano maggiore considerazione.
Per esempio, senza l'estrazione delle «terre rare», che avviene spesso in condizioni di sfruttamento del lavoro in alcune aree del mondo, non ci sarebbe sviluppo software né elettronico. I data center che alimentano e ospitano servizi online tramite torri di schede grafiche contribuiscono all'emissione di significative quantità di CO$_2$ nell'atmosfera e a un notevole consumo energetico. Inoltre, la gestione di queste strutture richiede un consumo intensivo di risorse naturali, tra cui miliardi di litri d'acqua all'anno, con gravi impatti sugli ecosistemi locali. Questi fattori pongono importanti questioni sulle conseguenze ambientali e sociali legate alla produzione e all'uso di tecnologie sempre più pervasive.

\chapter{Le terre rare}

\section{Scoperta e applicazioni}

La scoperta delle terre rare risale al 1787, quando una «dura pietra nera» venne trovata nella miniera della città svedese di Ytterby. Il chimico finlandese Johan Gadolin (1760-1852) non fu in grado di identificare la pietra e la chiamò «terra rara». La ricerca e l'isolamento di nuovi materiali portò alla scoperta di 17 elementi chimici (metalli) classificati come terre rare, o, in inglese, \textit{Rare Earth Elements} (REE). 

L'Unione internazionale di chimica pura e applicata (IUPAC) classifica le REE in:
\begin{itemize}
    \item Terre Rare Leggere (LREE): lantanio, cerio, praseodimio e neodimio;
    \item Terre Rare Intermedie (MREE): samario, europio, gadolinio, terbio, disprosio e olmio;
    \item Terre Rare Pesanti (HREE): erbio, tulio, itterbio e luterzio; le più rare nel suolo terrestre.
\end{itemize}

Nonostante il nome, le terre rare sono collettivamente più abbondanti nella crosta terrestre rispetto a metalli tradizionalmente considerati rari, come argento, piombo e oro. Il termine "raro" si riferisce in realtà alla difficoltà di estrazione e lavorazione, piuttosto che alla loro effettiva scarsità.
Negli anni 70 del secolo scorso si iniziarono a sfruttare le capacità magnetiche delle terre rare per creare supermagneti, fondamentali per creare motori elettrici.
Oggi le REE sono fondamentali per le tecnologie di comunicazione e informazione, grazie alle loro eccellenti proprietà come semiconduttori. Negli smartphone sono impiegate nella realizzazione di altoparlanti e motori di vibrazione (Nd e Dy), schermi e LED (Eu, Tb). Vengono utilizzate anche come catalizzatori in componenti di fibre ottiche e nei veicoli ibridi, dove svolgono un ruolo cruciale nel miglioramento delle prestazioni e nell'efficienza energetica.

\section{Miniere e geopolitica}
La relativa disponibilità di terre rare nel suolo non implica la possibilità di realizzare un supermagnete o una batteria con facilità. Anzi, l'intero processo che va dall'estrazione alla produzione finale è complesso, costoso e richiede tecnologie avanzate, oltre a porre importanti sfide ambientali e geopolitiche.

La Cina e gli Stati Uniti sono stati i primi a gettarsi nel mercato delle terre rare, attorno al 1950. L'infrastruttura per l'estrazione cinese si è particolarmente sviluppata tra il 1980 e 1990, grazie a riforme che hanno promosso lo sviluppo tecnologico e industriale, avendo già compreso l' utilità delle REE nel breve e nel lungo periodo. Gli Stati Uniti invece hanno assunto un ruolo marginale nell'estrazione ed esportazione di terre rare, probabilmente a causa di limitazioni legali nella gestione di sostanze altamente inquinanti, in quanto La loro lavorazione, fra le altre cose, rilascia prodotti di scarto tossici e radioattivi.

Oggi la Cina estrae, raffina e vende circa l'80-90\% delle terre rare globali. Questa posizione di dominio non è dovuta solamente alla lungimiranza del governo centrale e alla disponibilità di strumenti e risorse economiche: sfruttamento della mano d'opera, bassi standard ambientali, miniere illegali e il mercato nero hanno portato a prezzi con i quali l'Occidente non è in grado di competere. 

Il 70-80\% delle terre rare è estratta in miniere illegali a Bayan-Obo, o nelle montagne di Jiangxi, nella Cina meridionale. Qui uomini, donne e spesso anche minori vengono sfruttati per scavare e schiacciare rocce in quello che sembra un gigantesco formicaio, sempre attivo. «Like a human anthill, the mountain was mined twenty-four hours a day, seven days a week» \citep[p. 24]{pitron2020rare}.
L'estrazione illegale è stata vietata dalle autorità cinesi nel 2016, con pesanti multe per i trasgressori. Inoltre, enormi scorte di metalli promesse a mercati esteri sono state sequestrate al porto di Canton, e dozzine di trafficanti sono stati incarcerati.
Nonostante ciò, alcuni minatori continuano l'attività illecita nei punti più inospitali e nascosti delle montagne. I materiali estratti vengono venduti al mercato nero che li raffina e poi esporta in tutto il mondo.

Per estrarre le terre rare vengono impiegati principalmente due metodi. Il primo consiste nello scavare lo strato superficiale del terreno e creare vasche di estrazione, dove si versano reagenti chimici per separare i metalli dalla terra. Queste vasche, se non propriamente isolate, possono contaminare le falde acquifere e compromettere sistemi idrici.
Il secondo metodo consiste nel trivellare il terreno e inserire tubi in PVC, attraverso i quali vengono inserite sostanze chimiche direttamente nel sottosuolo. Anche in questo caso si crea una vasca di estrazione, con i medesimi effetti negativi del primo metodo. Inoltre, spesso questi tubi in PVC vengono abbandonati nel sito dello scavo. Entrambi i metodi presentano un forte impatto ambientale dovuto all'impiego di sostanze tossiche con drammatiche conseguenze anche sulla salute dei minatori.

Gli effetti negativi dello scavo delle REE non sono limitati al sottosuolo terrestre; coinvolgono anche:

\begin{itemize}
\item fiumi
\begin{quote}
\small
In 2006, some sixty companies producing indium — a rare metal used in the manufacture of certain solar panel technologies — released tonnes of chemicals into the Xiang River in Hunan, jeopardising the meridional province's drinking water and the health of its residents. \citep[p. 25]{pitron2020rare}
\end{quote}

\item coltivazioni ed ecosistemi
\begin{quote}
\small
'Men and women, wearing no more than basic face masks, work in areas thick with black particles and acid fumes. It's hell.' To complete the picture are toxic pits of chemical discharges from the plants, fields of poisoned corn, acid rain, and more. \citep[p. 26]{pitron2020rare}
\end{quote}

\item cittadini
\begin{quote}
\small
Rare earths have cost the community dearly. The hair of young men barely thirty years of age has suddenly turned white. Children grow up without developing any teeth. In 2010, the Chinese press reported that sixty-six Dalahai residents had died of cancer. \citep[p. 29]{pitron2020rare}
\end{quote}
    
\end{itemize}

\section{Interesse di Trump}

Gli Stati Uniti dipendono dalla Cina per molti minerali essenziali e terre rare utilizzati nelle tecnologie moderne, tra cui sistemi di difesa avanzati, industria aerospaziale, energia rinnovabile e produzione manifatturiera.
Il monopolio cinese sulle terre rare potrebbe essere uno dei motivi che ha recentemente spinto il presidente degli Stati Uniti, Donald Trump, a rivolgere l'attenzione verso paesi ricchi di questi materiali strategici, in particolare la Groenlandia e l'Ucraina.

Trump sostiene che il controllo sulla Groenlandia fornirebbe non solo maggiore ricchezza e indipendenza per gli Stati Uniti, ma anche un notevole vantaggio militare, installando missili balistici e radar sul fondo oceanico per rilevare navi russe e cinesi.
Gli Stati Uniti hanno colto l'importanza strategica della Groenlandia fin dal secondo dopoguerra, quando nel 1946 il presidente Truman provò a comprare l'isola dai danesi. Il 28 marzo 2025, il vicepresidente JD Vance ha visitato la Groenlandia, affermando che il governo locale ha «investito troppo poco nel popolo della Groenlandia e trascurato anche l'architettura della sicurezza di questa terra straordinaria, meravigliosa e popolata da persone eccezionali» \cite{foxnews2024greenland}. 
Le risorse minerarie dell'isola risultano ancora in gran parte inesplorate, dato che molte di esse sono sepolte sotto uno spesso strato di ghiaccio, e si stima che siano più abbondanti di quelle presenti in Cina. Costruire miniere per estrarre terre rare potrebbe essere la chiave per gli Stati Uniti per non dover più importare REE dalla Cina.
Tuttavia in Groenlandia sono presenti numerose leggi di tutela ambientale e sociale che renderebbero la costruzione di nuove miniere un processo lungo e costoso. Inoltre sono già state emanate leggi che vietano l'estrazione di gas e petrolio presente nel fondo oceanico attorno all'isola.

Il 30 aprile scorso USA e Ucraina hanno siglato il cosiddetto "mineral deal", un accordo economico che istituisce un fondo congiunto di investimento per la ricostruzione del Paese devastato dalla guerra e garantisce agli americani l'accesso alle risorse naturali ucraine, minerali, petrolio e gas. Trump ha voluto questa intesa a tutti i costi per presentarla agli americani come pagamento degli aiuti militari statunitensi ricevuti dagli ucraini dall'inizio del conflitto con la Russia. In realtà, l'Ucraina non dovrà ripagare nessun debito, ma contribuirà alla creazione di un fondo equamente diviso al 50\% e su cui entrambe le parti avranno pari diritti di voto.
Un'analisi condotta dal ministero ucraino della protezione ambientale e delle risorse naturali sostiene che nel paese si trovano depositi di 22 dei 50 materiali identificati come critici dagli Stati Uniti. Grafite, litio, titanio, berillio e uranio sono i minerali e metalli più presenti, fondamentali per l'industria high tech e la transizione energetica. Le terre rare ucraine sono in gran parte inutilizzate a causa delle politiche statali, della mancanza di informazioni attendibili sui giacimenti e della guerra in corso. Tre dei sei giacimenti sono concentrati nelle zone più colpite, contese o occupate dall'esercito russo.
Senza un'intesa di pace tra Ucraina e Russia, lo sviluppo e l'attivazione del fondo restano quindi di fatto irrealizzabili. L'accordo sulle risorse minerarie rafforza ulteriormente la posizione di Trump come potenziale intermediario credibile per avviare un dialogo tra Kiev e Mosca, oggi del tutto assente.

\section{Riciclo e recupero}
Attualmente il riciclo delle terre rare è limitato da una raccolta inefficace, alti costi di smontaggio e dall'assenza di metodi economicamente sostenibili per il recupero.
La piccola quantità che viene effettivamente riciclata proviene da:
\begin{itemize}
    \item magneti permanenti usati in pale eoliche;
    \item elementi utilizzati come catalizzatori nell'industria chimica;
    \item fosfori per l'illuminazione nelle lampade fluorescenti, poiché in molte giurisdizioni queste lampade devono essere raccolte per isolare il mercurio pericoloso al loro interno;
    \item magneti utilizzati nei dischi rigidi dei server.
\end{itemize}

In seguito all'improvviso aumento dei prezzi delle terre rare esportate dalla Cina tra il 2010 e il 2011, molti ricercatori hanno avviato studi per trovare metodi più efficaci per il recupero e il riutilizzo delle REE.
Nel 2014 Allan Walton, presso l'Università di Birmingham, ha sviluppato un processo di riciclo dei magneti al neodimio che consente di recuperare 1 kg di materiale consumando solo il 12\% dell'energia necessaria per l'estrazione di una quantità equivalente da materie prime, con una riduzione del 98\% della tossicità per l'uomo. La procedura prevede l'esposizione dei magneti all'idrogeno per poi essere assorbito dal neodimio, consentendo successivamente di separare, tramite setacciatura, la polvere di neodimio-ferro-boro dal rivestimento in nichel.
Attualmente, sono in corso ricerche in Giappone, presso l'Università di Tohoku, nell'\textit{European Rare Earth (Magnet) Recycling Network} e nel \textit{Critical Materials Institute} negli Stati Uniti.

\chapter{L'ascesa delle GPU}
La produzione e diffusione delle schede grafiche (GPU) ebbe inizio alla fine degli anni '70, con la realizzazione di dispositivi hardware per visualizzare testo sui monitor dei personal computer. Negli anni '90 e 2000, aziende come NVIDIA e ATI (oggi AMD) cominciarono a sviluppare GPU capaci di eseguire il rendering 3D, sfruttando architetture basate su silicio e incrementando il numero di transistor per core. Il loro vantaggio rispetto alle CPU risiede nella capacità di eseguire simultaneamente numerosi compiti semplici attraverso l'uso di molteplici core in parallelo. Fino al 2010, le GPU venivano utilizzate principalmente per la loro capacità di rappresentare oggetti 3D in modo preciso, soprattutto nei videogiochi e nelle applicazioni professionali.

\section{Produzione e impatto ambientale}

I materiali grezzi utilizzati da NVIDIA, l'azienda leader nella produzione di GPU con una quota di mercato globale pari a circa l'80\%, includono alluminio, lamine di rame, fibre di vetro, gel di silicone, stagno, titanio e tungsteno. NVIDIA è consapevole e dichiara pubblicamente che «stagno, titanio e tungsteno sono "minerali da conflitto"\citep[p.36]{nvidia2021report}, legati a violenze e violazioni dei diritti umani»; molti di questi materiali sono importati da Cina e Indonesia.
Il consumo di elettricità da parte di NVIDIA per la gestione delle proprie strutture è la loro fonte principale di gas serra, con circa 68.000 kg di emissioni. L'energia totale richiesta è aumentata del 33\% dal 2019-2020, e quasi metà di questa è impiegata nelle operazioni nei loro data center.

Oltre a utilizzare 1/3 di energia rinnovabile dal 2020, NVIDIA afferma di aver ridotto i loro flussi di rifiuti liquidi inviati a discariche del 78\%. Stanno anche facendo sforzi per implementare misure di conservazione idriche. Tuttavia, il loro fornitore TSMC usa 157,000 tonnellate d'acqua minerale al giorno per produrre semiconduttori. Non utilizzano acqua riciclata poiché rischierebbe di contaminare l'elettronica.

I sistemi computazionali basati su architetture parallele presentano un impatto ambientale rilevante: sebbene ogni singola GPU abbia un consumo energetico inferiore rispetto a una CPU, l'elevato numero di unità impiegate ne annulla i benefici in termini di efficienza. A ciò si aggiungono i costi energetici legati ai sistemi di raffreddamento e le conseguenti emissioni di gas serra associate al funzionamento dell'intero sistema.

\section{Utilizzi}

\subsection{AI e machine learning}

La capacità delle GPU di eseguire più compiti in parallelo ha cambiato il modo in cui le applicazioni di analisi e processo dei dati sono strutturate. Fornisce ad algoritmi per AI e machine learning la velocità ed efficienza necessaria per eseguire calcoli complessi.
In alcune GPU moderne sono presenti core specializzati per l'elaborazione di operazioni sui tensori, fondamentali nel deep learning. Questi consentono di accelerare le moltiplicazioni tra matrici, migliorando sia l'efficienza di addestramento che l'inferenza delle reti neurali.

I costi elevati dell'infrastruttura necessaria al funzionamento di ChatGPT sono il principale motivo per cui, quando la startup cinese DeepSeek ha lanciato un modello di intelligenza artificiale con costi di mantenimento nettamente inferiori rispetto a quelli di OpenAI, le azioni di NVIDIA, principale fornitore di GPU per l'addestramento e l'esecuzione di tali modelli,  hanno subito un brusco calo, comportando una perdita del valore di mercato vicina a \$600mld.

Negli ultimi anni si è visto lo sviluppo di GPU ottimizzate per l'AI: un trend destinato a crescere in futuro, visti i continui progressi che si stanno facendo su quel fronte.

\subsection{Mining di criptovalute}

Nell'ultimo decennio, una varietà di valute digitali chiamate criptovalute si sono diffuse nel mercato mondiale. L'estrazione, o "mining", di criptovalute consiste nel raccogliere un insieme di transazioni effettuate negli ultimi dieci minuti all'interno di un blocco e risolvere un complesso problema matematico associato ad esso. Il primo miner che riesce a risolvere il problema aggiunge il blocco alla blockchain (una specie di registro digitale che contiene informazioni in modo sicuro e verificabile), rendendo ufficiali le transazioni contenute, e riceve come ricompensa un premio in Bitcoin.

Per risolvere questi problemi inizialmente erano impiegati gruppi di CPU, per poi essere sostituite da GPU. La crescente popolarità delle criptovalute, e il conseguente consumo energetico richiesto per il mining produce enormi quantità di anidride carbonica. Inoltre, il rapido sviluppo di GPU sempre più performanti per il mining rende i modelli precedenti rapidamente obsoleti, producendo notevoli rifiuti elettronici (E-Waste). Una gestione inadeguata di questi rifiuti può provocare il rilascio di sostanze nocive nel suolo e contribuire all'inquinamento atmosferico.

L'impronta idrica del mining di criptovalute è una questione ambientale sempre più rilevante. Per raffreddare i gruppi computer utilizzati nelle operazioni di mining, si impiegano grandi quantità di acqua. Solamente per l'estrazione di Bitcoin, la valuta più vecchia e popolare, sono stati utilizzati 1,600 gigalitri d'acqua in un anno: una singola transazione consuma tanta acqua quanto quella contenuta in una piscina domestica. Il consumo elettrico annuale è paragonabile a quello del Portogallo, e viene prodotta una quantità di E-Waste simile a quella dei Paesi Bassi.

Visti gli alti costi idrici ed elettrici per alimentare questi computer, sono stati presi provvedimenti legali per limitare l'impatto ambientale: nel 2021, la Cina ha vietato completamente il mining di Bitcoin, nel 2022 lo stato di New York ha vietato centri di mining alimentati da energia fossile. In Canada, la ditta elettrica del Quebec ha aumentato i prezzi e limitato l'utilizzo massimo di energia destinata al mining. Nel 2022 la commissione europea ha invitato gli stati membri a ridurre il consumo di elettricità dei miner.

\subsection{Veicoli a guida autonoma}

La sicurezza dei veicoli a guida autonoma è strettamente legata alla loro capacità computazionale e alla rapidità con cui possono prendere decisioni in tempo reale. Le GPU, grazie alla loro architettura parallela, sono fondamentali per gestire simultaneamente compiti complessi come l'identificazione di oggetti e l'analisi delle immagini, ottimizzando così le prestazioni dei sistemi di percezione e reazione del veicolo. NVIDIA fornisce servizi per lo sviluppo di sistemi di guida autonoma basati su modelli AI addestrati su moltissimi scenari di guida, testati e validati in uno spazio virtuale, NVIDIA Omniverse.

Il problema del surriscaldamento delle GPU si presenta anche qua: una macchina con GPU a bordo deve disporre di un notevole sistema di raffreddamento. Queste due componenti, assieme a continui aggiornamenti software, contribuiscono all'elevato prezzo delle auto a guida autonoma.

\subsection{Data center}

\begin{quote}
\small
In the twenty-first century [...], the technology needed for turning simple activities into recorded data became increasingly cheap; and the move to digital-based communications made recording exceedingly simple.\citep[p.28]{srnicek2017platform}
\end{quote}

Con la diffusione delle architetture client-server e la crescente domanda di sistemi decentralizzati, i data center si sono evoluti da semplici elaboratori di grandi dimensioni a infrastrutture iper-scalari, distribuite strategicamente su scala globale per garantire prestazioni elevate e bassa latenza agli utenti in tutto il mondo.

All'interno dei data center, le GPU coprono un ruolo cruciale nella gestione di calcoli complessi, algoritmi di machine learning e supporto per simulazioni scientifiche altamente precise. L'importanza delle GPU nei data center è destinata a crescere assieme alla quantità e complessità di dati che le aziende raccolgono. Un report dell'\textit{International Energy Agency} afferma che i data center nel 2021 consumavano circa l'1-1.5\% dell'energia mondiale. Questi dati risalgono a prima del boom dell'intelligenza artificiale e, secondo la banca d'investimento Goldman Sachs, una singola query a ChatGPT richieda quasi 10 volte l'elettricità di una ricerca su Google. Per questo prevede che la domanda di energia dei data center crescerà del 160\% entro il 2030.

Secondo un report di The Guardian\cite{guardian2024datacenters}, tra le aziende del gruppo GAFAM, i data center di Amazon risultano essere i maggiori consumatori di energia, richiedendo più del doppio dell'elettricità rispetto alla seconda classificata, Apple. Un singolo data center di Amazon attualmente consuma l'equivalente elettrico di 50.000 case. Per questo motivo le GAFAM in primo piano, ma anche altri data center, sono sotto pressione dai governi locali per diventare carbon neutral e passare a energie rinnovabili.

\chapter{Consumo d'acqua nei data center}

\begin{quote}
\small
As a recorded entity, any datum requires sensors to capture it and massive storage systems to maintain it. Data are not immaterial, as any glance at the energy consumption of data centres will quickly prove. [...] Altogether, this means that the collection of data today is dependent on a vast infrastructure to sense, record, and analyse.\citep[p.28]{srnicek2017platform}
\end{quote}

Le GPU costituiscono solo una parte di questa infrastruttura, che deve rimanere costantemente operativa. A essa si aggiungono router, hard disk, server e numerose altre apparecchiature di rete, tutte componenti che contribuiscono in modo significativo all'elevato consumo energetico. Si pensi a Uber o ad altre piattaforme lean che per espandere la propria attività non hanno bisogno di acquistare nuove automobili, ma semplicemente di potenziare la propria infrastruttura digitale, ad esempio affittando più server. \citep[p.31]{srnicek2017platform}

Inoltre, esistono data center di aziende esterne all'settore informatico che necessitano anch'esse di enormi data center. Solo il business del gas naturale liquefatto della General Electric raccoglie una quantità di dati paragonabile a quella di Facebook, e richiede strumenti altamente specializzati per gestire un flusso informativo di tale portata. \citep[p.40]{srnicek2017platform}

A causa dell'enorme consumo energetico di queste strutture, la quantità d'acqua utilizzata per il raffreddamento del loro hardware è in realtà inferiore se paragonata a quella richiesta per la produzione di elettricità. Il processo di produzione di elettricità si basa sul riscaldamento dell'acqua che, trasformandosi in vapore, mette iQuesti dati risalgono da prima del boom dell'intelligenza artificiale, secondo la banca d'investimento Goldman Sachs, una singola query a ChatGPT richieda quasi 10 volte l'elettricità di una ricerca su Google.n movimento una turbina e genera energia elettrica.

I data center di grandi dimensioni e strutture iper-scalari spesso gestiscono autonomamente la propria produzione di energia per garantire un'elevata continuità operativa (o \textit{uptime}), soprattutto in situazioni di emergenza, come nel caso di interruzioni della rete elettrica pubblica dovute a blackout.
Essi utilizzano sistemi a ciclo diretto che prelevano l'acqua da corpi idrici vicini; l'acqua, riscaldata dal vapore prodotto nel condensatore, viene poi restituita alla fonte originale, ma a una temperatura superiore. Questi sistemi sono particolarmente vulnerabili e pericolosi in periodi di siccità e caldo estremo, poiché l'acqua restituita può superare i limiti termici consentiti, con impatti negativi sugli ecosistemi locali.
Per ridurre questi impatti, alcuni data center hanno adottato sistemi di raffreddamento a ciclo chiuso, che riciclano le acque di scarico e piovane, riducendo così l'uso di acqua fresca del 50-70\%. In questo tipo di sistema, l'acqua di raffreddamento non viene restituita alla sorgente, ma convogliata in torri di raffreddamento, dove il calore viene disperso tramite evaporazione. Il resto dell'acqua viene quindi ricircolato attraverso i condensatori.

Alcuni data center scelgono di localizzare le proprie strutture in aree strategiche per ottimizzare il raffreddamento: zone montane dove l'aria è naturalmente più fresca, o sott'acqua, come ha fatto Microsoft in Scozia, dove l'acqua fredda dell'oceano viene utilizzata per raffreddare le apparecchiature.

Attualmente, gli Stati Uniti sono il paese con il maggior numero di data center, con più di 5300 sempre in funzione. I quantitativi di acqua consumati da una singola struttura sono difficili da stimare, poiché non sono di dominio pubblico e spesso oggetto di discussione, dato l'impatto negativo che possono avere sulle comunità locali in cui vengono costruiti. Si ipotizza che un data center di piccole dimensioni consumi circa 26 milioni di litri d'acqua all'anno tra raffreddamento e produzione di elettricità, e che Google abbia utilizzato 5 miliardi di litri solamente per il raffreddamento delle sue strutture.

\section{Conseguenze dell'uso di acqua minerale e potabile}

Con l'espansione dei servizi online, accelerata dal lockdown dovuto alla pandemia di Covid-19, si è registrato un aumento significativo nella costruzione di data center, spesso senza considerare adeguatamente le conseguenze economiche e ambientali legate alla posizione geografica scelta.
Uno studio fatto da Landon Marston, professore della Virginia Tech, afferma che l'industria dei data center statunitense preleva il 90\% della sua acqua da bacini idrogeologici, e che il 20\% di questi siano sovrasfruttati. Sono stati fatti sforzi da parte della big tech per impiegare acqua non potabile: Google afferma che il 25\% delle risorse idriche da loro impiegate derivino da prodotti fognari trattati. In media, queste alternative all'acqua potabile costituiscono solo il 5\% del consumo totale di un data center. 

Si consideri il caso della città di Mesa, in Arizona, dove nel maggio 2021 è stato approvato lo sviluppo di un data center di \$800mln di dollari in una zona in alta allerta per la siccità, la più alta registrata negli ultimi 126 anni. Il vicesindaco Jenn Duff ha affermato che «[...] i data center sono un uso irresponsabile della nostra acqua» \cite{nbc2021drought}. Duff fa parte del numero crescente di individui che esprimono preoccupazioni circa la diffusione dei data center sul territorio nazionale, soprattutto in zone già aride, dove l'energia solare ed eolica è ampiamente disponibile e a prezzi bassi. 
Simili preoccupazioni hanno portato ad un divieto temporaneo alla costruzione di nuovi data center nei Paesi bassi e alla promulgazione di leggi per maggiore trasparenza in Francia.

La costruzione di data center è promossa anche nelle città dei paesi in via di sviluppo per accelerare la trasformazione digitale, offrendo incentivi alle big tech in cambio di investimenti e accesso ai mercati emergenti. Questo avviene senza considerare le conseguenze per il territorio e per le comunità locali, che spesso subiscono impatti negativi.
In India, si stima che il data center di Bengaluru consumi 8 milioni di litri d'acqua al giorno, minacciando i fragili sistemi idrici di una città che ha già vissuto una grave siccità.
Inoltre, durante la peggiore siccità registrata negli ultimi 74 anni, il governo dell'Uruguay ha concesso a Google l'autorizzazione per la costruzione di data center, suscitando proteste tra la popolazione locale.

\section{Impatto sugli ecosistemi}
Il fabbisogno idrico dei data center è destinato a crescere, soprattutto con l'espansione di siti dedicati all'addestramento di sistemi di intelligenza artificiale. Si prevede che, a breve, il consumo d'acqua legato all'industria dell'AI supererà quello di settori tradizionalmente ad alto consumo idrico, come l'allevamento di bestiame e l'industria tessile. Inoltre l'impronta di CO$_2$ è maggiore di quella del settore aereo.

I già citati sistemi di raffreddamento a ciclo diretto, fortunatamente ormai in disuso, erano responsabili della morte di miliardi di pesci all'anno, intrappolando nei loro sistemi di assunzione anche uova e larve. L'inquinamento termico provoca un aumento del metabolismo della fauna locale, con conseguente incremento del loro fabbisogno alimentare. Questo, a sua volta, porta a una carenza di cibo, spingendo le specie a migrare verso habitat più sostenibili. Il riscaldamento dell'acqua provoca anche una riduzione dei livelli di ossigeno dissolto nell'acqua, il quale è fondamentale per la sopravvivenza di organismi in ambienti acquatici.

Questi problemi non sono limitati ai sistemi di raffreddamento dei data center, ma di tutti gli impianti geotermici.

\section{Alternative sostenibili}

I data center più estesi hanno già provato a mettere in atto soluzioni sostenibili. Oltre al caso già citato di Microsoft in Scozia, il data center di Google ad Hamina, in Finlandia, utilizza acqua marina per il raffreddamento sin dalla sua inaugurazione nel 2011. Anche Amazon, seppur con scarsi risultati, si sta impegnando per essere meno dipendente dall'acqua minerale.

\subsection{Reattori modulari e fusione}
Per ridurre il consumo idrico e garantire un uptime prossimo al 100\%, molte big tech stanno investendo in fonti energetiche alternative. Sebbene le energie rinnovabili come l'eolico e il solare offrano il vantaggio di zero emissioni, non sono in grado di assicurare una fornitura continua di elettricità altamente ambita dalle big tech. Per questo motivo, l'area che sta suscitando maggiore interesse e investimenti è quella della fusione nucleare, che mira a produrre energia in modo continuo e senza emissioni.

Gli Stati Uniti hanno prodotto una minuscola quantità di energia tramite fusione nucleare nel 2022, al Lawrence Livermore National Ignition Facility. Da allora investimenti nel settore sono saliti fino a \$8mld. Di particolare rilievo è l'investimento da \$1mld effettuato da Sam Altman, CEO di OpenAI, in Helion, azienda specializzata nella ricerca sulla fusione nucleare. Questa iniziativa è motivata dall'elevato fabbisogno energetico necessario per sostenere infrastrutture come quella di ChatGPT, il cui addestramento e funzionamento costante comporta costi energetici estremamente elevati.

Dopo il crollo di fiducia nella fissione nucleare seguito al disastro di Fukushima (Giappone) nel 2011, negli Stati Uniti si sta assistendo ad una vera e propria "rinascita" del settore. Amazon ha firmato accordi con Energy Northwest per la costruzione di piccoli reattori modulari (SMR) in grado di produrre l'energia necessaria al sostegno di 770.000 case. Anche Google ha annunciato un accordo con Kairos Power per produrre 500 megawatt di elettricità tramite SMR entro il 2035.
Ha suscitato grande clamore l'annuncio di Microsoft, che in collaborazione con Constellation Energy ha deciso di riattivare l'impianto nucleare di Three Mile Island, oggetto nel 1979 di un parziale meltdown che portò al rilascio di gas radioattivi nell'atmosfera.

\chapter{Conclusione}

In questa tesina sono stati evidenziati i principali effetti negativi derivanti dalla diffusione eccessiva di prodotti e infrastrutture tecnologiche, con particolare attenzione all'impatto sull'ecosistema globale e sull'umanità, in qualità sia di consumatrice sia di produttrice di tecnologia.

Ero già consapevole della grande quantità di materie prime necessaria per la produzione di dispositivi tecnologici, ma non mi aspettavo che anche il consumo idrico fosse così elevato, soprattutto in aree già colpite da scarsità d'acqua. 
Lo sfruttamento dei lavoratori, anche minori, e dei territori mi ha molto colpito. La comunità internazionale e gli Stati nazionali dovrebbero porre più attenzione all'impatto sociale ed ambientale dell'estrazione delle terre rare.
Nonostante queste criticità, rimango ottimista riguardo alle innovazioni tecnologiche in corso, atte a migliorare i processi di riciclo e ridurre il consumo energetico del settore. Confido che tali soluzioni possano essere implementate prima che gli effetti negativi sull'ambiente diventino irreversibili.
Mi auguro, inoltre, che le più grandi aziende tecnologiche riescano a raggiungere al più presto la neutralità carbonica, privilegiando, tuttavia, fonti di energia sostenibili e rinnovabili, come quella solare, eolica e geotermica; in altre parole, ritengo doveroso per una società civile privilegiare la salute dell'uomo salvaguardando l'ambiente. La transizione verso un sistema energetico sostenibile presenta certamente molte sfide come la necessità di affrontare le fluttuazioni energetiche delle fonti sostenibili, migliorare le tecnologie di stoccaggio dell'energia e affrontare gli aspetti economici e politici legati alla transizione.
Tuttavia, queste sfide offrono anche opportunità per il progresso tecnologico, la collaborazione internazionale e la creazione di nuove politiche pubbliche a sostegno di queste tecnologie. Investire nelle fonti rinnovabili e adottare pratiche energetiche sostenibili non solo ci aiuterà a proteggere l'ambiente, ma garantirà anche un futuro più stabile ed equo per le generazioni future.

\newpage

\chapter*{Fonti}
\renewcommand{\bibsection}{}
\section*{Riferimenti bibliografici e citazioni}
\nocite{*}

\bibliography{refs}
\section*{Paper scientifici}
\subsection*{Capitolo 1}
\urlstyle{same}
\sloppy

\url{https://www.sciencedirect.com/science/article/abs/pii/S0959652612006932} \\
\url{https://www.sciencedirect.com/science/article/abs/pii/S0176268022001598} \\
\url{https://www.gssc.lt/wp-content/uploads/2025/02/v04_Boruta_Rare-earths_A4_EN.pdf} \\
\url{https://www.sciencedirect.com/science/article/abs/pii/S2452223617301256} \\
\url{https://www.sciencedirect.com/science/article/pii/S0301420725000613}\\
\url{https://aldoagostinelli.com/terre-rare}

\subsection*{Capitolo 2}
\url{https://neptjournal.com/upload-images/(55)B-4152.pdf}

\subsection*{Capitolo 3}
\url{https://www.nrdc.org/sites/default/files/power-plant-cooling-IB.pdf}\\
\url{https://www.nature.com/articles/s41545-021-00101-w}

\section*{Siti web e articoli di giornale}
\subsection*{Capitolo 1}
\url{https://hir.harvard.edu/not-so-green-technology-the-complicated-legacy-of-rare-earth-mining/} \\
\url{https://www.bbc.com/news/articles/c9d5jwvw9nlo} \\
\url{https://www.soci.org/chemistry-and-industry/cni-data/2016/2/recyling-rare-earths}
\url{https://www.renewablematter.eu/cosa-prevede-accordo-ucraina-stati-uniti-risorse-minerarie}


\subsection*{Capitolo 2}
\url{https://retropcparts.com/gpu-manufacturing-process/} \\
\url{https://www.designlife-cycle.com/nvidia-gpu} \\
\url{https://tecex.com/gpu-powerhouses-behind-ai/} \\
\url{https://www.perplexity.ai/page/the-gpu-shortage-explained-ori-7BswhHKvT_idmwUL0P845Q} \\
\url{https://telnyx.com/resources/gpu-architecture-ai} \\
\url{https://medium.com/@vikaskumar8449/understanding-the-resources-consumed-by-ai-models-and-their-environmental-impact-b4fd9b2168d7} \\
\url{https://www.soprasteria.com/insights/details/what-is-the-actual-environmental-cost-of-ai} \\
\url{https://semianalysis.com/2024/03/13/ai-datacenter-energy-dilemma-race/} \\
\url{https://www.techtarget.com/searchdatacenter/feature/How-the-rise-in-AI-impacts-data-centers-and-the-environment} \\
\url{https://en.wikipedia.org/wiki/Environmental_impact_of_bitcoin} \\
\url{https://www.investopedia.com/tech/gpu-cryptocurrency-mining/} \\
\url{https://www.nvidia.com/en-us/self-driving-cars/} \\
\url{https://www.icdrex.com/gpus-for-self-driving-cars-powering-the-automotive-industrys-potential/} \\
\url{https://www.intel.com/content/www/us/en/products/docs/discrete-gpus/data-center-gpu/what-is-data-center-gpu.html} \\
\url{https://www.trgdatacenters.com/resource/the-role-and-purpose-of-data-center-gpus/}
\url{https://www.reuters.com/technology/chinas-deepseek-sets-off-ai-market-rout-2025-01-27/}

\subsection*{Capitolo 3}
\url{https://www.asce.org/publications-and-news/civil-engineering-source/civil-engineering-magazine/} \\
\url{https://dgtlinfra.com/data-center-cooling/} \\
\url{https://dgtlinfra.com/data-center-water-usage/} \\
\url{https://eng.ox.ac.uk/case-studies/the-true-cost-of-water-guzzling-data-centres/} \\
\url{https://www.lawfaremedia.org/article/ai-data-centers-threaten-global-water-security} \\
\url{https://www.weforum.org/stories/2024/11/circular-water-solutions-sustainable-data-centres/}\\
\url{https://time.com/6240746/nuclear-fusion-breakthrough-milestone-clean-energy}\\
\url{https://www.cnbc.com/2024/12/28/why-microsoft-amazon-google-and-meta-are-betting-on-nuclear-power.html}\\
\url{https://www.forbes.com/sites/greatspeculations/2024/10/21/why-tech-giants-are-betting-big-on-nuclear-power/}\\
\url{https://en.wikipedia.org/wiki/Three_Mile_Island_accident}
\newpage
\end{document}
